\documentclass[10pt,serif,draftclsnofoot,onecolumn]{IEEEtran}
\usepackage{color}
\usepackage{setspace}
\usepackage{url}
\singlespacing

\newcommand{\HRule}[1]{\rule{\linewidth}{#1}}
\begin{document}

\begin{titlepage}
  \title{ \normalsize \textsc{}
    \\ [2.0cm]
    \HRule{0.5pt} \\
    \LARGE \textbf{\uppercase{Tech Review}}
    \\ \normalsize \textsc{the arliss project}
    \HRule{2pt} \\ [0.5cm]
    \normalsize \today \vspace*{5\baselineskip}}
  \date{11/12/2016}
  
  \author{Paul Minner \\
    Oregon State University \\
    Senior Capstone}
  \pagenumbering{gobble}	
  \maketitle
\end{titlepage}

\newpage
\pagenumbering{arabic}

\section{Parachute Deployment}

\subsection{Options}
The first option being considered for deploying the parachute is deploying at a specific altitude based on an altitude sensor. This option would require software to be written to read the values sent from an altitude sensor. The second option is deploying the parachute after a certain amount of time. This would require a timer to implement. The last, and simplest solution is to deploy the parachute immediately after the satellite separates from the rocket. This option could be entirely mechanical, and therefore not require any software to implement.
\subsection{Goals for Use}
In order for the parachute deployment to be considered a success, it must activate before hitting the ground. Ideally, the satellite would reach the ground as quickly as safely possible in order to maximize the battery life of the satellite.
\subsection{Evaluation Criteria}
These options will be evaluated based on simplicity, time taken to hit the ground, accuracy, power draw. Simplicity is important to see if a slightly better solution is actually worth the effort to implement. The time taken to hit the ground has an effect on how much power is used before the satellite starts moving towards it’s target and is therefore important. Accuracy is important since we don’t get a second chance if the satellite is destroyed. Finally, power draw is important since power is a limited resource.

\begin{table}[h!]
  \centering
  \caption{Comparison of evaluation criteria for parachute deployment.}
  \label{tab:table1}
  \begin{tabular}{l||l|l|l|l}
                         & Simplicity & Time    & Accuracy & Power\\
    \hline
    Altitude Sensor      & Least      & Fastest & High     & High\\
    \hline
    Timer                & Less       & Fast    & Lower    & High\\
    \hline
    Immediate Deployment & Most       & Slow    & High     & Low\\
  \end{tabular}
\end{table}

\subsection{Discussion}
The altitude sensor solution will most likely be the most accurate, and take the least amount of time to hit the ground, but may draw more power than other solutions, and is the least simplistic. This is because software has to be written to check the value of a sensor incrementally, which is the most complicated solution, and constantly checking the value of a sensor will drain power faster than if there were no software. The timer based solution will be less accurate, and draw more power than the final solution, but will take less time to hit the ground, and is simpler than the first solution. There is still the same problem about power, which is that a timer has to be checked constantly and therefore drains power. The final solution to deploy the parachute immediately is the simplest solution, since it requires more software, but it will take much longer for the satellite to reach the ground than the other two solutions. The accuracy of this solution should be excellent as well.

\subsection{Selection}
We have selected the timer based solution. This is because we need the satellite to reach the ground relatively quickly to save power. We believe the power draw from the software to check the timer would be less than the power lost from being stuck in the air for much longer if we had picked the immediate deployment method. The problem with the altitude sensor option is that we currently aren’t planning to add an altitude sensor, and space is a premium, since the rover must fit in a soda can.
	
\section{Getting Unstuck from Obstacles}

\subsection{Options}
Our three options we’re considering involve either adding another hardware component, using existing servos, or just using the treads. The first option is to add a mechanical arm which deploys to get the satellite unstuck from whatever it hit. The second option is to attempt moving different directions with the treads while moving the treads up and down with the servos they are attached to. The third option is to simply use the treads on their own and attempt moving different directions until the satellite can move. 

\subsection{Goals for Use}
The goal of this system is to recover the satellite if the obstacle avoidance system fails and the satellite hits an obstacle. This is a difficult goal since we don’t expect the obstacle avoidance system to fail, and if it does, the rover has encountered something we didn’t anticipate. Hopefully, this system is never actually used. 

\subsection{Evaluation Criteria}
These options will be evaluated on simplicity, likelihood of success, and space constraints. Simplicity is important since more effort may not be worth a slight improvement over another option. Likelihood of success is difficult to guess, but obviously it is very important since the solution needs to work. Space constraints are important since space is limited on the satellite.

\begin{table}[h!]
  \centering
  \caption{Comparison of evaluation criteria for getting unstuck from obstacles.}
  \label{tab:table1}
  \begin{tabular}{l||l|l|l}
                                 & Simplicity & Success      & Space \\
    \hline
    Mechanical Arm               & Least      & Most Likely  & Uses Space  \\
    \hline
    Moving Tread Servos          & Less       & Likely       & Doesn't Use Space \\
    \hline
    Moving Different Directions  & Most       & Least Likely & Doesn't Use Space  \\
  \end{tabular}
\end{table}

\subsection{Discussion}
The mechanical arm option is the most likely to succeed since it could implement everything the other options implement as well, but it is by far the most complicated and uses the most space. Determining the best placement of the arm and how to use it also isn’t obvious. The arm would be most useful when the treads are completely stuck, because the other options wouldn’t be able to deal with this. The option which raises and lowers the treads would be the next most complicated, but it may be more likely to succeed than the last option. Without testing, however, it’s difficult to determine how helpful changing the height of the treads would actually be in getting unstuck. This option doesn’t require any extra space, either. The last option to simply move the treads in different directions to get unstuck is the simplest solution, and requires no space. This option is also the least likely to work, since all other options implement this as well as something else. 

\subsection{Selection}
We will use the simplest solution, which just attempts to move in different directions until the satellite can move. We picked this solution because it is the simplest to implement, and we weren’t sure how well the other solutions would actually work. The mechanical arm option would take up too much space, so it’s not an option, but the option which raises and lowers the treads would have worked. We just don’t know if changing the height of the treads would actually help get the satellite unstuck, so for now we aren’t including it. If, in testing, we find that our simple solution doesn’t work well, we may implement the ability to raise and lower the treads as well.

\section{Find and Touch the Finish Pole}

\subsection{Options}
Our options are based on either brute force, or sensors. The first option is to drive in a square pattern, like a lawnmower, in the general area of the pole until we hit it. The second option is to drive in a circular pattern outward until we hit the pole. The final option is to use sensors to detect where the pole is, and drive straight into it. 

\subsection{Goals for Use}
Our goal is to hit the pole as quickly as possible. We must hit the pole to finish the competition, so this system needs to work correctly, otherwise we lose the competition. Speed is important to preserve battery life. The satellite will likely be low on power at this point, so finishing before it runs out of power is important.

\subsection{Evaluation Criteria}
We will evaluate these options based on speed, simplicity and accuracy. Speed is important to make sure we don’t run out of power. Simplicity is important to save us time. Finally, accuracy is important because if we fail to hit the pole, we lose the competition.

\begin{table}[h!]
  \centering
  \caption{Comparison of evaluation criteria for finding and touching the finish pole.}
  \label{tab:table1}
  \begin{tabular}{l||l|l|l}
                          & Speed & Simplicity   & Accuracy \\
    \hline
    Square Pattern        & Slow  & Simple       & More Accurate  \\
    \hline
    Circular Pattern      & Slow  & Simple       & More Accurate \\
    \hline
    Detect using Sensors  & Fast  & Complicated  & Less Accurate  \\
  \end{tabular}
\end{table}

\subsection{Discussion}
A separate mode of finding the finish pole is needed because GPS isn’t accurate enough. GPS is accurate to within a few meters, but the pole will likely only be a couple inches wide. The first option solves this problem by getting to where the GPS says the pole is an searching in a square pattern, like you would run a lawn mower. This is very simple to implement, and should be accurate. Unfortunately, it may take a long time to find the pole. The second option is just like the first, except the satellite searches in a circular pattern, instead of a square pattern. This has the same pros and cons as the first option. The last option is to search for the pole using the sensors used for the obstacle avoidance system, then driving straight into the pole. This option is the most complicated, and may be less accurate than the other solutions, but could be much faster.

\subsection{Selection}
We chose to go with the circular search pattern. We chose this solution because a circular pattern may take slightly less time than a square pattern, since the satellite is checking the area closest to the starting destination, rather than checking far away corners using a square pattern. The option to use sensors doesn’t seem accurate enough to us. We may change to the sensor approach later on if when we test the sensors, they seem very accurate.

\newpage
\end{document}
