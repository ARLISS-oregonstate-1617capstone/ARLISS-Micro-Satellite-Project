\documentclass[10pt,letterpaper,onecolumn,draftclsnofoot,journal]{IEEEtran}
\usepackage[margin=0.75in]{geometry}
\usepackage{listings}
\usepackage{color}
\usepackage{longtable}
\usepackage{graphicx}
\usepackage{float}
\usepackage{tabu}
\usepackage{enumitem}
\usepackage{courier}
\usepackage[hidelinks]{hyperref}

\setlength{\parindent}{0cm}

\begin{document}
\begin{titlepage}
	\title{The ARLISS Project\\Progress Report\\CS 461}
	\author{Steven Silvers, Paul Minner, Zhaolong Wu, Zachary DeVita\\
		Capstone Group 27, Fall 2016}
	\date{\today}
	\maketitle
	\vspace{4cm}
	\begin{abstract}
		\noindent Abstract goes here...
	\end{abstract}

\end{titlepage}
\tableofcontents
\clearpage

\section{Introduction}
This document is a progress report which details our team's progress on the ARLISS Project throughout the Fall term of 2016. It  contains a week by week summary of all related activities, problems we have encountered, solutions to prior problems, and a detailed retrospective of the last ten weeks. The retrospective will contain a section for things which have gone well, a section for changes which need to be implemented, and a section which will detail actions which will need to occur to make those changes.  

\section{Summary of the Term}
\subsection{Week 3}
By week 3 our team was established, we had met with our mentor and client for the first time, and our team had also met with our extended team of electrical and mechanical engineers. In class we had been briefly described upcoming assignments for our project, as well as, had been assigned the 'Problem Statement'. Our project, the ARLISS Project, is to build a satellite, i.e. robot, and compete in the ARLISS competition. ARLISS stands for "A Rocket Launch for International Student Satellites," and it is a national competition where teams comprised of multiple engineering disciplines build a "soda-can" sized satellite which must autonomously navigate to a specified location after being ejected from a rocket at approximately 12,000' AGL and landing safely on the ground.
 
The first assignment, as mentioned prior, was the 'Problem Statement'. The problem statement is a document which was written after meeting with our team's client, and it contains a definition of the problem, for which our team is providing a software solution for, and it includes some of the specific details for our project as prescribed by the client. This document is the predecessor to the 'Requirements Document' and is meant to server as a "10,000 ft." view of the project with less of a focus on the specifics or how to implement the software. 

At the end of the third week our team had not yet finished the 'Problem Statement'. Our team initially had some difficulty meeting with our client and our extended team, i.e. the mechanical and electrical engineers. The assignment due date was pushed back a few days, but it was nearly finished at that point anyway. 

We met with our client near the end of the third week. Her input was extraordinarily limited which has proven to be both good and bad. Our client is completely leaving all decisions regarding the project up to us. This allows us to have complete freedom for the project, and it allows us to not need to meet with her very often, but it also leaves us with no expertise or direction for the project. With all things considered, I think this may result in more of a hindrance than a benefit.

\subsection{Week 4}


\subsection{Week 5}


\subsection{Week 6}


\subsection{Week 7}


\subsection{Week 8}
\par
Week eight was highlighted by making final edits to and submitting our technology review for grading. Once the technology review was done and behind us we began focusing on the next assignment, the design document. Work on the design document in week eight was focused around planning out how to approach the document and the overall outline for what it will look like.
\par
We held our all-group meeting with the mechanical and electrical engineering sub-teams on Tuesday night where we further discussed the hardware implementation of our project. Some of the decisions made included using GPS to calculate altitude for parachute deployment, using a 9V 1200~1500 mA battery for the main power supply, as well as using two motors for driving. It is still undecided as to what kind of sensor will be used for detecting obstacles, whether it will be an image sensor or SONAR/RADAR. This raises a problem because the decision as to what kind of sensor to use directly impacts how our design document will be written and if our technology review will need revision.
\par
The plan for week nine is to write the design document and finish it early so we have more time to focus on the final progress report and presentation. There will be another all-group meeting where we will hopefully finalize our decision as to what kind of sensor we will be using to detect and avoid obstacles. We are still brainstorming how we would like to approach the progress report in the following weeks, whether we will record as a group or individually and then splice our presentations together.


\subsection{Week 9}
\par
A lot of work done during week nine was done individually as it fell on the Thanksgiving holiday break. Most of the team finished their individual portions for the design document over break, which was fairly easy to do without much collaboration as each person was responsible for writing up the same sections they wrote about in the technology review. We held our usual all-group meeting Tuesday night, where further discussion was had in regards to sensors and other design aspects of the project. It was decided that a CMOS imaging sensor would be used to detect objects in the satellite's path.
\par
It is still uncertain what kind of control board the electrical team will implement for the project. It currently looks like they are between using a Raspberry PI Zero and building their own custom board with an ATMega128 chip. This decision will hopefully be made soon as ultimately effects how our overall system will be designed and implemented.
\par
The plan for week ten is to work on the progress report early, and finish the presentation the weekend before finals week. We decided to meet as a group in a study room on campus Sunday afternoon to record the audio and screen capture for the presentation. It was decided that each person would be responsible for a portion of the progress report, and then to prepare their own speaking portions for the presentations and arrive Sunday ready to record. Steven volunteered his Engr webspace for hosting the project submission.

\subsection{Week 10}



\section{A Reflection of Development Period}
\begin{tabular}{ |p{0.3\linewidth}|p{0.3\linewidth}|p{0.3\linewidth}|  }
\hline
\multicolumn{3}{|c|}{Retrospective} \\
\hline
Positives& Deltas &Actions \\
\hline
placeholder & placeholder &placeholder \\
placeholder & placeholder &placeholder \\
placeholder & placeholder &placeholder \\
\hline
\end{tabular}

\section{Conclusion}
Conclusion goes here...

%\section{References}

%\bibliographystyle{IEEEtran}
%\bibliography{tech}

\end{document}
