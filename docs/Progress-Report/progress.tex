\documentclass[10pt,letterpaper,onecolumn,draftclsnofoot,journal]{IEEEtran}
\usepackage[margin=0.75in]{geometry}
\usepackage{listings}
\usepackage{color}
\usepackage{longtable}
\usepackage{graphicx}
\usepackage{float}
\usepackage{tabu}
\usepackage{enumitem}
\usepackage{courier}
\usepackage[hidelinks]{hyperref}

\setlength{\parindent}{0cm}

\begin{document}
\begin{titlepage}
	\title{The ARLISS Project\\Progress Report\\CS 461}
	\author{Steven Silvers, Paul Minner, Zhaolong Wu, Zachary DeVita\\
		Capstone Group 27, Fall 2016}
	\date{\today}
	\maketitle
	\vspace{4cm}
	\begin{abstract}
		\noindent Abstract goes here...
	\end{abstract}

\end{titlepage}
\tableofcontents
\clearpage

\section{Introduction}
This document is a progress report which details our team's progress on the ARLISS Project throughout the Fall term of 2016. It  contains a week by week summary of all related activities, problems we have encountered, solutions to prior problems, and a detailed retrospective of the last ten weeks. The retrospective will contain a section for things which have gone well, a section for changes which need to be implemented, and a section which will detail actions which will need to occur to make those changes.  

\section{Summary of the Term}
\subsection{Week 3}
By week 3 our team was established, we had met with our mentor and client for the first time, and our team had also met with our extended team of electrical and mechanical engineers. In class we had been briefly described upcoming assignments for our project, as well as, had been assigned the 'Problem Statement'. Our project, the ARLISS Project, is to build a satellite, i.e. robot, and compete in the ARLISS competition. ARLISS stands for "A Rocket Launch for International Student Satellites," and it is a national competition where teams comprised of multiple engineering disciplines build a "soda-can" sized satellite which must autonomously navigate to a specified location after being ejected from a rocket at approximately 12,000' AGL and landing safely on the ground.
 
The first assignment, as mentioned prior, was the 'Problem Statement'. The problem statement is a document which was written after meeting with our team's client, and it contains a definition of the problem, for which our team is providing a software solution for, and it includes some of the specific details for our project as prescribed by the client. This document is the predecessor to the 'Requirements Document' and is meant to server as a "10,000 ft." view of the project with less of a focus on the specifics or how to implement the software. 

At the end of the third week our team had not yet finished the 'Problem Statement'. Our team initially had some difficulty meeting with our client and our extended team, i.e. the mechanical and electrical engineers. The assignment due date was pushed back a few days, but it was nearly finished at that point anyway. 

We met with our client near the end of the third week. Her input was extraordinarily limited which has proven to be both good and bad. Our client is completely leaving all decisions regarding the project up to us. This allows us to have complete freedom for the project, and it allows us to not need to meet with her very often, but it also leaves us with no expertise or direction for the project. With all things considered, I think this may result in more of a hindrance than a benefit.

\subsection{Week 4}


\subsection{Week 5}


\subsection{Week 6}


\subsection{Week 7}


\subsection{Week 8}


\subsection{Week 9}


\subsection{Week 10}



\section{A Reflection of Development Period}
\begin{tabular}{ |p{0.3\linewidth}|p{0.3\linewidth}|p{0.3\linewidth}|  }
\hline
\multicolumn{3}{|c|}{Retrospective} \\
\hline
Positives& Deltas &Actions \\
\hline
placeholder & placeholder &placeholder \\
placeholder & placeholder &placeholder \\
placeholder & placeholder &placeholder \\
\hline
\end{tabular}

\section{Conclusion}
Conclusion goes here...

%\section{References}

%\bibliographystyle{IEEEtran}
%\bibliography{tech}

\end{document}
