\documentclass[10pt,serif,draftclsnofoot,onecolumn]{IEEEtran}
\usepackage{color}
\usepackage{setspace}
\usepackage{url}
\singlespacing

\newcommand{\HRule}[1]{\rule{\linewidth}{#1}}
\begin{document}
	\begin{titlepage}


	\title{ \normalsize \textsc{}
			\\ [2.0cm]
			\HRule{0.5pt} \\
			\LARGE \textbf{\uppercase{Tech Review}}
			\\ \normalsize \textsc{the arliss project}
			\HRule{2pt} \\ [0.5cm]
			\normalsize \today \vspace*{5\baselineskip}}
	\date{10/16/2016}
	
	\author{Steven Silvers \\
			Oregon State University \\
			Senior Capstone}
	\pagenumbering{gobble}	
	\maketitle
	\end{titlepage}
	\newpage
	\pagenumbering{arabic}
	\section{Control Board}
	\subsection{Options}
	The options being considered for the control board in our project are the Arduino Uno, Rasperry Pi Model A+ and a custom board developed by a team of ECE capstone students.
	\subsection{Goals for Use}
	The control board is what brings all the hardware together to create a single, functioning system. Our software system will be loaded directly onto this board to control input and output to the various system peripherals, such as the motors and GPS module.
	\subsection{Evaluation Criteria}
	These board options will be evaluated based on their ability to run the embedded system that we develop, ability to interface with various required hardware sensors of the system, if they meet the space constraints of the CanSat design, and how well they manage limited resources such as power.
	\subsection{Discussion}
	\paragraph{Arduino Uno}
	The Arduino Uno is the base level Arduino board and is based upon the ATmega328P board with fourteen digital I/O pins, six of which support pulse with modulation. Arduino supports its own language based on Java, C and C++ as well as its own IDE making it easy to work with. The Uno is 68.6mm long by 53.4mm wide, which means it will fit in a soda can with dimensions 66.3mm by 115.2mm however, we do not know how much of that space will be taken up by other hardware. Battery wise the Uno only requires a maintained five volts from a battery to operate. The Uno is a traditional microcontroller, once it is given power it immediately begins running code which may be ideal for our project.
	\paragraph{Raspberry Pi Model A+}
	The model A+ is what is recommended by Raspberry Pi to use for embedded projects due to its lower power consumption. The Raspberry Pi measures 85.60mm by 56mm by 21mm which may make it a very tight fit based on the CanSat space constraints. Another point to mention is that Raspberry Pis aren't traditional microcontrollers, but are instead full fledged computers. The Pi also has a few extra I/O devices prepackaged that aren't necessary for our project, such as an HDMI port.
	\paragraph{Custom Board}
	A custom built board would be the best option if done correctly. The dimensions could be developed with the Can Sat restrictions in mind and could only include the necessary hardware. On the downside it would be a untested board, with no way of knowing how much power it would consume or if it would be able to run our system until after it is designed, built and tested.
	\subsection{Selection}
	With these pros and cons in mind, I would select the Arduino Uno as the controller for our project. It is a popular, well documented board that we know for sure works. My second choice would be the custom board, only because with the given time line of the project if it didn't work we would be in a very bad place. The Pi comes in third due mostly to size, not being an actual microcontroller and extra devices that are not needed.
	
\section{Avoiding Obstacles}
\subsection{Options}
	Options for obstacle avoidance include detecting obstacles and then driving out of the way to get around it, develop an algorithm to determine if it can be driven over or not, or drive straight through the obstacle.
\subsection{Goals for Use}
	The goal of this system is to make sure our CanSat can reach the finish destination without getting caught or stuck on anything in the CanSat's way.
\subsection{Evaluation Criteria}
	The technology will be evaluated on how easy it will be to develop, and how effective it will be at accomplishing the goal of reaching the destination. 
\subsection{Discussion}
\paragraph{Drive Around}
	This method of avoiding obstacles while seeming like the smart choice has some problems. Driving around an obstacle would be smart if the obstacle were say a large rock, but if it were something like a long ditch created by a car tire it could go on for miles, and our CanSat would never be able to reach the destination. It could be fairly easy to develop, as soon as it detected an obstacle it could run a routine that makes the CanSat backup, turn and move forward a bit before trying to proceed to the goal.
\paragraph{No Avoidance}
	This method while sounding awful could actually work. Our CanSat will be tested in the Black Rock Desert of Nevada, a very flat, open space. It would be very easy to develop this as it would mean no programming, which would also save memory on the board. However, if any obstacles at all popped it could be game over for our CanSat.
\paragraph{Algorithm Avoidance}
	This would be the most difficult of the options to implement and would involve heavy testing as well as a great knowledge of automated driving systems. On the up side, knowing when the CanSat can and cannot get over or past an obstacle would be extremely helpful, cutting down drive time and therefore saving battery.
\subsection{Selection}
	With the evaluation criteria in mind, developing an algorithm to determine whether or not the CanSat should try to avoid an obstacle is the best option for accomplishing our goal. While the other two options would be much easier, they also carry much higher risk of the CanSat failing to reach the finish point, making the algorithm based avoidance the responsible choice.
\section{Mode of Transportation}
\subsection{Options}
	The options considered for mode of transportation are a traditional two wheel setup with a wheel at the top and the bottom of the CanSat and the CanSat being oriented so that the can body would by horizontal. The next option would be using caterpillar tracks with the can oriented vertically. The final option would be a four wheel setup with the can oriented horizontally.
\subsection{Goals for Use}
	The goal for this part of the system is the actual mode of transportation for getting the CanSat from where it landed to its final destination.
\subsection{Evaluation Criteria}
	This technology will be evaluated on its ability to efficiently transport the CanSat, ability to handle rough terrain, and ability to conserve space within the delivery CanSat.
\subsection{Discussion}
\paragraph{Two Wheels}
	The two wheel setup would most likely use the least amount of power out of the three options, as there would be two total motors. Since they would be oriented horizontally, you could get the biggest wheels possible with the size constraints of the can. Bigger wheels would be better for getting over certain obstacles as opposed to small wheels. With the horizontal orientation, there would not be very much ground clearance between the CanSat body and the ground, which could raise a problem with very small obstacles. Because it is oriented on only two wheels, getting flipped on becomes a not very big concerned because the wheels would always be touching the ground.
\paragraph{Four wheels}
	The four wheel option would provide a stable base for driving the CanSat, but also opens up the CanSat for possibly getting flipped by an obstacle and stuck. This option could be done with two or four motors, but with size limitations already being pushed with four wheels, it would most likely be limited to two motors.
\paragraph{Caterpillar Tracks}
	This option is interesting, as the tracks would definitely be the best at traversing rough terrain and getting over obstacles. On the downside they would take up a lot more room in the CanSat than traditional wheels and would also require a bit more power.
\subsection{Selection}
	Based on the above criteria and the pros and cons of each, I would use the caterpillar tracks. What they give up in size consumption and power consumption, they more than make up for in ability to get over rough terrain. It does not matter how much space or power is saved for other devices if our CanSat gets stuck on an obstacle while driving to the goal.
	\newpage



\end{document}