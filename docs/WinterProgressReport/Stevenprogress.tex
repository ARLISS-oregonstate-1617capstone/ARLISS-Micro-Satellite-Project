\documentclass[10pt,letterpaper,onecolumn,draftclsnofoot,journal]{IEEEtran}
\usepackage[margin=0.75in]{geometry}
\usepackage{listings}
\usepackage{color}
\usepackage{longtable}
\usepackage{graphicx}
\usepackage{float}
\usepackage{tabu}
\usepackage{enumitem}
\usepackage{courier}
\usepackage[hidelinks]{hyperref}

\setlength{\parindent}{0cm}

\begin{document}
\begin{titlepage}
	\title{The ARLISS Project\\Progress Report\\CS 461}
	\author{Steven Silvers, Paul Minner, Zhaolong Wu, Zachary DeVita\\
		Capstone Group 27, Fall 2016}
	\date{\today}
	\maketitle
	\vspace{4cm}
	\begin{abstract}
		\noindent This document details our experiences designing our project in the last ten weeks. This includes progress made, problems encountered, and plans for the future. We also detail changes we plan to implement in the future.
	\end{abstract}

\end{titlepage}
\tableofcontents
\clearpage

\section{Introduction}
This section details the progress made by team member Steven Silvers during the first half of winter term  as well as any work done over winter break. Steven is responsible for making sure the control board for our project will provide what we need, and for determining what kind of wheel system our rover will use. The biggest piece that Steven is responsible for is the obstacle avoidance module, which is the main tool used by our autonomous driving system to keep our rover from becoming stuck during the competition.

\section{Summary of the Term}
\subsection{Week 1}
Start of winter term, over break I studied various autonomous driving methods since that is the main part of the project I am responsible for. I also looked at different hardware implementations and how they would possible effect our programming decisions. We are communicating with the rest of the ARLISS capstone team to find a weekly meeting time, and have set our weekly TA meeting for Tuesdays at 1:30pm. Hopefully our whole team meeting time will be set soon so the ECE and ME teams can update us on their progress.

\subsection{Week 2}
Week two we met with our TA Franks for the first time this term, and discussed what is expected of the group as far as implementation of our project for this term. We as a group decided what a successful alpha version of our project would look like. We decided since the other two teams on our project would mostly likely not be finished with the hardware implementation until the end of the term, that for our alpha release we would write simulators for our individual modules to demonstrate that they function properly. The goal for the Beta release is to have the code implemented on hardware, but that is entirely dependent on the progress made by the ME and ECE groups.


\subsection{Week 3}
Week three we held a team wide meeting with the ECE and ME groups to better figure out time lines and what everyone is working on. It sounds like the ECE team has finally settled on using a Raspberry Pi zero as the main board for the rover, giving us a better idea of how we need to implement our code. The ME team told us that they will be designing and creating the wheel system in house so that it is fully customized to meet our needs. Progress has continued to be made on developing and simulating our individual code pieces in preparation for the week 6 alpha release.

\subsection{Week 4}
This week was mostly business as usual, we continued working on the alpha version of our project, as well as had an all group meeting that included our client Dr. Squires. My development module has been changed in that it will be getting a .mat file as input instead of a 2d binary array. This change will need to be reflected in the requirements document. As it is a late change, the alpha version will still be making use of a 2D binary array, and the plan is to change over to the .mat file for the beta version.

\subsection{Week 5}
Week five was spent doing work on the alpha version of the modules I am responsible for, mostly the obstacle avoidance system. Because of this modules' complexity it was decided that the alpha would be a lower level "prof of concept" and full functionality would be added by the 1.0 release. I edited both the tech review and the design document to reflect changes that had been made to the project for the sections I am responsible for. The biggest edit was for the control board, reflecting the decision made by the ECE team to use a Raspberry Pi Zero instead of one of the previously listed options.

\subsection{Week 6}
Week six was focused entirely on the upcoming alpha release and progress report. We decided that each team member would write their pieces individually and then combine them into one document Thursday, giving us plenty of time to submit the report on Friday. We got together with the ECE and ME teams for our weekly meeting, which ended early because all three groups were busy with progress reports and had nothing of significance to share. The ME team is uncertain of their capstone number, which hopefully they will figure out soon so that we can register as a group for Expo.

\section{Conclusion}
Overall, our project so far has been very successful. There have been some difficulties, such as being unsure of what hardware components were being used for so long, and realizing the hardware for the rover won't be completed until the end of winter term, but we have yet to encounter any problems which we can't find a solution to. We have a plan for how to design each component of our software, and if we implement the actions mentioned in the table above, creating our project should go very smoothly.

%\section{References}

%\bibliographystyle{IEEEtran}
%\bibliography{tech}

\end{document}
