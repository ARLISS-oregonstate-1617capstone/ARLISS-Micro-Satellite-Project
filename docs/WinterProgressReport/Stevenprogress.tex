\documentclass[10pt,letterpaper,onecolumn,draftclsnofoot,journal]{IEEEtran}
\usepackage[margin=0.75in]{geometry}
\usepackage{listings}
\usepackage{color}
\usepackage{longtable}
\usepackage{graphicx}
\usepackage{float}
\usepackage{tabu}
\usepackage{enumitem}
\usepackage{courier}
\usepackage[hidelinks]{hyperref}

\setlength{\parindent}{0cm}

\begin{document}
\begin{titlepage}
	\title{The ARLISS Project\\Progress Report\\CS 461}
	\author{Steven Silvers, Paul Minner, Zhaolong Wu, Zachary DeVita\\
		Capstone Group 27, Fall 2016}
	\date{\today}
	\maketitle
	\vspace{4cm}
	\begin{abstract}
		\noindent This document details our experiences designing our project in the last ten weeks. This includes progress made, problems encountered, and plans for the future. We also detail changes we plan to implement in the future.
	\end{abstract}

\end{titlepage}
\tableofcontents
\clearpage

\section{Introduction}
This document is a progress report which details our team's progress on the ARLISS Project throughout the Fall term of 2016. It  contains a week by week summary of all related activities, problems we have encountered, solutions to prior problems, and a detailed retrospective of the last ten weeks. The retrospective will contain a section for things which have gone well, a section for changes which need to be implemented, and a section which will detail actions which will need to occur to make those changes.  

\section{Summary of the Term}
\subsection{Week 3}
By week 3 our team was established, we had met with our mentor and client for the first time, and our team had also met with our extended team of electrical and mechanical engineers. In class we had been briefly described upcoming assignments for our project, as well as, had been assigned the Problem Statement. Our project, the ARLISS Project, is to build a satellite, i.e. robot, and compete in the ARLISS competition. ARLISS stands for "A Rocket Launch for International Student Satellites," and it is a national competition where teams comprised of multiple engineering disciplines build a "soda-can" sized satellite which must autonomously navigate to a specified location after being ejected from a rocket at approximately 12,000' AGL and landing safely on the ground.\vspace{.3cm}
\par
The first assignment, as mentioned prior, was the Problem Statement. The problem statement is a document which was written after meeting with our team's client, and it contains a definition of the problem, for which our team is providing a software solution for, and it includes some of the specific details for our project as prescribed by the client. This document is the predecessor to the Requirements Document and is meant to serve as a "10,000 ft." view of the project with less of a focus on the specifics or how to implement the software.\vspace{.3cm}
\par 
At the end of the third week our team had not yet finished the Problem Statement. Our team initially had some difficulty meeting with our client and our extended team, i.e. the mechanical and electrical engineers. The assignment due date was pushed back a few days, but it was nearly finished at that point anyway.\vspace{.3cm}
\par 
We met with our client near the end of the third week. Her input was extraordinarily limited which has proven to be both good and bad. Our client is completely leaving all decisions regarding the project up to us. This allows us to have complete freedom for the project, and it allows us to not need to meet with her very often, but it also leaves us with no expertise or direction for the project. With all things considered, I think this may result in more of a hindrance than a benefit.

\subsection{Week 4}
The beginning of week four was spent finishing up the rough draft of the Problem Statement, getting it signed by our client, and turning it in. We managed to finish the rough draft by our extended due date without any additional issues. The team seems to work well together; everyone does their part, we are able to keep in excellent communication, and everyone seems more than capable of meeting their deadlines.\vspace{.3cm}
\par 
In addition to turning in the final draft of the Problem Statement, we also met had a meeting with our mentor on Monday of the week. We were able to speak with him about changes that would need to be made in our Problem Statement, as well as, briefly discuss some of the upcoming assignments. He was able to help clarify some of the confusion in the Problem Statement assignment description and point us in the right direction\vspace{.3cm}
\par. 
We also met with our extended team on Tuesday to discuss the project. The requirements for the competition are very open-ended so we spent most of the meeting trying to narrow down our team's specific design for the project. A requirement for the competition is that the satellite autonomously navigates to the target destination after in lands back on Earth, but there is no requirement for whether the satellite flies or drives on the ground. This is one of the major decisions made at this meeting. We decided for our satellite to autonomously drive to its destination. This was primarily due to the fact that our satellite will only weigh approximately 150 grams and even the slightest wind in the wrong direction would prevent it from reaching the target. This and the fact that a quadcopter would require much more energy to power itself, and battery power is our most critical resource for our project. We discussed the other components for the project like programming languages and design of the satellite, but nothing else was set in concrete.\vspace{.3cm}
\par   
For the remainder of the week, our team worked on incorporating some of the changes recommended by our mentor, as well as, the changes recommended by our instructors. Much of the formatting for the document was not finished in the prior week so this was a major focus for the week as well. As for plans for the following week, our team had discussed the Requirements document briefly, as a preparatory measure. The final draft of the Problem Statement would be turned in early in the next week, and the Requirements document would be the primary focus for the rest of the week. 

\subsection{Week 5}
Our team wrapped up the Problem Statement over the weekend, and we got it signed by our client on Monday morning.  The final draft of the assignment was finished and turned in on time without any issues. Again, our team met with our mentor on Monday morning, as well as, our extended team on Tuesday evening. For this week, the primary goal was to finish a rough draft of the Requirements document.\vspace{.3cm}
\par
When we met with our mentor we discussed the Requirements document. There was a bit of confusion in regards to this assignment to say the least. He was able to answer a few of the questions we had, but this only served to marginally dilute the confusion.
Our team had a considerably difficult time planning for this assignment. This document is a contract between the engineers and the client and it is intended to strictly define the tasks which our team is intended to complete. Our team, with collaboration from the extended team of engineers who are additionally working on this project, have been given full reign for the design and implementation of the project. We are solely responsible for the entire design of the project with no outside input or expertise. This means we, as a team, had to essentially develop our own requirements based on a few vague competition guidelines from their website and some educated “assumptions.”\vspace{.3cm}
\par 
Even knowing we had to develop our own requirements did not suffice. The hardware for our project had not been determined at this point and the guidelines for the competition were not available. The hardware for our project will not be available until some point in the middle of winter term. This greatly limited our team's ability to write a realistic, legitimate set of requirements. For our circumstances it seems that it would have been far more advantageous to write this document after the research had been done, but our team also understands that not all team circumstances are the same as ours.\vspace{.3cm}
\par
Either way, our team discussed the assignment in detail. We were able to brainstorm ideas for the document and delegate sections of the assignment to each of the members in our team so that we could complete the assignment. We managed to turn in a rough draft towards the end of the week. After the rough draft was turned in, the class as a whole was instructed by Dr. Kevin McGrath to spend the weekend revising our rough drafts and turning in an alternate version of the rough draft. Apparently there were enough teams equally as confused as us over the expectations of the document that it was necessary to give some clarification to the entire class and an extension on the rough draft.\vspace{.3cm}
\par 
The plan for the following week was to implement the changes which were recommended by our instructors for the rough draft of the Requirements document, and then work on the final draft of the assignment.


\section{Conclusion}
Overall, our project so far has been very successful. There have been some difficulties, such as being unsure of what hardware components were being used for so long, and realizing the hardware for the rover won't be completed until the end of winter term, but we have yet to encounter any problems which we can't find a solution to. We have a plan for how to design each component of our software, and if we implement the actions mentioned in the table above, creating our project should go very smoothly.

%\section{References}

%\bibliographystyle{IEEEtran}
%\bibliography{tech}

\end{document}
